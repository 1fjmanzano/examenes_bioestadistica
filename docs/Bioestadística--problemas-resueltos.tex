% Options for packages loaded elsewhere
\PassOptionsToPackage{unicode}{hyperref}
\PassOptionsToPackage{hyphens}{url}
%
\documentclass[
]{book}
\usepackage{lmodern}
\usepackage{amsmath}
\usepackage{ifxetex,ifluatex}
\ifnum 0\ifxetex 1\fi\ifluatex 1\fi=0 % if pdftex
  \usepackage[T1]{fontenc}
  \usepackage[utf8]{inputenc}
  \usepackage{textcomp} % provide euro and other symbols
  \usepackage{amssymb}
\else % if luatex or xetex
  \usepackage{unicode-math}
  \defaultfontfeatures{Scale=MatchLowercase}
  \defaultfontfeatures[\rmfamily]{Ligatures=TeX,Scale=1}
\fi
% Use upquote if available, for straight quotes in verbatim environments
\IfFileExists{upquote.sty}{\usepackage{upquote}}{}
\IfFileExists{microtype.sty}{% use microtype if available
  \usepackage[]{microtype}
  \UseMicrotypeSet[protrusion]{basicmath} % disable protrusion for tt fonts
}{}
\makeatletter
\@ifundefined{KOMAClassName}{% if non-KOMA class
  \IfFileExists{parskip.sty}{%
    \usepackage{parskip}
  }{% else
    \setlength{\parindent}{0pt}
    \setlength{\parskip}{6pt plus 2pt minus 1pt}}
}{% if KOMA class
  \KOMAoptions{parskip=half}}
\makeatother
\usepackage{xcolor}
\IfFileExists{xurl.sty}{\usepackage{xurl}}{} % add URL line breaks if available
\IfFileExists{bookmark.sty}{\usepackage{bookmark}}{\usepackage{hyperref}}
\hypersetup{
  pdftitle={Bioestadística: problemas resueltos},
  pdfauthor={Javier Manzano},
  hidelinks,
  pdfcreator={LaTeX via pandoc}}
\urlstyle{same} % disable monospaced font for URLs
\usepackage{longtable,booktabs}
\usepackage{calc} % for calculating minipage widths
% Correct order of tables after \paragraph or \subparagraph
\usepackage{etoolbox}
\makeatletter
\patchcmd\longtable{\par}{\if@noskipsec\mbox{}\fi\par}{}{}
\makeatother
% Allow footnotes in longtable head/foot
\IfFileExists{footnotehyper.sty}{\usepackage{footnotehyper}}{\usepackage{footnote}}
\makesavenoteenv{longtable}
\usepackage{graphicx}
\makeatletter
\def\maxwidth{\ifdim\Gin@nat@width>\linewidth\linewidth\else\Gin@nat@width\fi}
\def\maxheight{\ifdim\Gin@nat@height>\textheight\textheight\else\Gin@nat@height\fi}
\makeatother
% Scale images if necessary, so that they will not overflow the page
% margins by default, and it is still possible to overwrite the defaults
% using explicit options in \includegraphics[width, height, ...]{}
\setkeys{Gin}{width=\maxwidth,height=\maxheight,keepaspectratio}
% Set default figure placement to htbp
\makeatletter
\def\fps@figure{htbp}
\makeatother
\setlength{\emergencystretch}{3em} % prevent overfull lines
\providecommand{\tightlist}{%
  \setlength{\itemsep}{0pt}\setlength{\parskip}{0pt}}
\setcounter{secnumdepth}{5}
\usepackage{booktabs}
\usepackage[spanish]{babel}
\decimalpoint
\selectlanguage{spanish}
\ifluatex
  \usepackage{selnolig}  % disable illegal ligatures
\fi
\usepackage[]{natbib}
\bibliographystyle{apalike}

\title{Bioestadística: problemas resueltos}
\author{Javier Manzano}
\date{2023-10-05}

\begin{document}
\maketitle

{
\setcounter{tocdepth}{1}
\tableofcontents
}
\hypertarget{introducciuxf3n}{%
\chapter{Introducción}\label{introducciuxf3n}}

En estas páginas encontrarás problemas resueltos tipo examen de Bioestadística para la asignatura en Grados de Ciencias de la Salud (Enfermería, Fisioterapia, Farmacia, etc.)

Estas páginas son un complemento al \href{https://1fjmanzano.github.io/bioestadistica/}{Curso de Bioestadística} que incluye prácticas con Excel©.

En temario sobre el que basamos esta colección de problemas es el de la asignatura de la Universidad de Salamanca que incluye los siguientes bloques temáticos:

\begin{itemize}
\item
  Planteamiento de una investigación: Anatomía y Fisiología de la investigación
\item
  Análisis Descriptivo y Gráfico de datos cuantitativos
\item
  Análisis Inferencial. Aplicaciones.
\item
  Regresión y correlación.
\item
  Tablas de contingencia.
\item
  Medidas de importancia clínica.
\end{itemize}

\hypertarget{planteamiento-de-una-investigaciuxf3n-anatomuxeda-y-fisiologuxeda-de-la-investigaciuxf3n}{%
\chapter{Planteamiento de una investigación: Anatomía y Fisiología de la investigación}\label{planteamiento-de-una-investigaciuxf3n-anatomuxeda-y-fisiologuxeda-de-la-investigaciuxf3n}}

En este capítulo se resolverán problemas relativos a:

\begin{itemize}
\tightlist
\item
  Diseño de una investigación
\item
  Métodos de muestreo
\item
  Métodos de recolección de datos
\item
  Variables y Escalas de Medida
\item
  Errores en la Investigación
\end{itemize}

\hypertarget{pregunta-test}{%
\section{Pregunta test}\label{pregunta-test}}

En una muestra de pacientes, el número de varones dividido entre el total de pacientes es:

\begin{enumerate}
\def\labelenumi{\alph{enumi})}
\tightlist
\item
  Una frecuencia relativa.
\item
  Una frecuencia absoluta.
\item
  Una variable cuantitativa.
\item
  Una variable cualitativa.
\item
  Un valor de la variable.
\end{enumerate}

\hypertarget{soluciuxf3n}{%
\subsection{Solución}\label{soluciuxf3n}}

\begin{enumerate}
\def\labelenumi{\alph{enumi})}
\tightlist
\item
  \href{https://1fjmanzano.github.io/bioestadistica/tablas-de-frecuencias.html}{Explicación}
\end{enumerate}

\hypertarget{pregunta-test-1}{%
\section{Pregunta test}\label{pregunta-test-1}}

Señale cuál de las siguientes afirmaciones es falsa:

\begin{enumerate}
\def\labelenumi{\alph{enumi})}
\tightlist
\item
  La aparición o no de bacterias en un cultivo es una variable dicotómica
\item
  La estatura de un individuo es una variable cuantitativa discreta.
\item
  El lugar que ocupa una persona entre sus hermanos (de menor a mayor edad) es una variable ordinal.
\item
  El estado civil es una variable cualitativa.
\item
  La glucemia es continua.
\end{enumerate}

\hypertarget{soluciuxf3n-1}{%
\subsection{Solución}\label{soluciuxf3n-1}}

\begin{enumerate}
\def\labelenumi{\alph{enumi})}
\setcounter{enumi}{1}
\tightlist
\item
  \href{https://1fjmanzano.github.io/bioestadistica/tipos-de-variables.html}{Explicación}
\end{enumerate}

\hypertarget{pregunta-test-2}{%
\section{Pregunta test}\label{pregunta-test-2}}

En el caso de una variable ordinal, el número n de datos válidos es:

\begin{enumerate}
\def\labelenumi{\alph{enumi})}
\tightlist
\item
  La suma de las frecuencias absolutas.
\item
  La frecuencia absoluta acumulada de la categoría más frecuente.
\item
  La suma de las frecuencias relativas.
\item
  La frecuencia relativa acumulada en la última catetgoría.
\item
  La (a) y la (d) son ciertas.
\end{enumerate}

\hypertarget{soluciuxf3n-2}{%
\subsection{Solución}\label{soluciuxf3n-2}}

\begin{enumerate}
\def\labelenumi{\alph{enumi})}
\tightlist
\item
  \href{https://1fjmanzano.github.io/bioestadistica/tablas-de-frecuencias.html}{Explicación}
\end{enumerate}

\hypertarget{pregunta-test-3}{%
\section{Pregunta test}\label{pregunta-test-3}}

En un estudio sobre problemas cervicales preguntamos a los pacientes acerca del tipo de almohada que usan. Las respuestas deberían ser consideradas como una variable:

\begin{enumerate}
\def\labelenumi{\alph{enumi})}
\tightlist
\item
  Cualitativa nominal
\item
  Numérica
\item
  Discreta
\item
  Continua.
\item
  Ordinal
\end{enumerate}

\hypertarget{soluciuxf3n-3}{%
\subsection{Solución}\label{soluciuxf3n-3}}

\begin{enumerate}
\def\labelenumi{\alph{enumi})}
\tightlist
\item
  \href{https://1fjmanzano.github.io/bioestadistica/tipos-de-variables.html}{Explicación}
\end{enumerate}

\hypertarget{pregunta-test-4}{%
\section{Pregunta test}\label{pregunta-test-4}}

Elija la afirmación correcta sobre variables observadas en individuos:

\begin{enumerate}
\def\labelenumi{\alph{enumi})}
\tightlist
\item
  Poseer vivienda propia es una variable numérica.
\item
  Poseer animales de compañía es una variable cualitativa.
\item
  La nacionalidad es una variable ordinal.
\item
  El tipo de almohada que usa es variable ordinal.
\item
  La longitud de la cama donde duerme es variable discreta.
\end{enumerate}

\hypertarget{soluciuxf3n-4}{%
\subsection{Solución}\label{soluciuxf3n-4}}

\begin{enumerate}
\def\labelenumi{\alph{enumi})}
\setcounter{enumi}{1}
\tightlist
\item
  \href{https://1fjmanzano.github.io/bioestadistica/tipos-de-variables.html}{Explicación}
\end{enumerate}

\hypertarget{pregunta-test-5}{%
\section{Pregunta test}\label{pregunta-test-5}}

La estadística en Ciencias de la Salud se utiliza para obtener información sobre situaciones de caracter:

\begin{enumerate}
\def\labelenumi{\alph{enumi})}
\tightlist
\item
  Determinista.
\item
  Sistemático.
\item
  Exhaustivo.
\item
  Aleatorio.
\item
  Excluyente.
\end{enumerate}

\hypertarget{soluciuxf3n-5}{%
\subsection{Solución}\label{soluciuxf3n-5}}

\begin{enumerate}
\def\labelenumi{\alph{enumi})}
\setcounter{enumi}{3}
\tightlist
\item
  \href{https://1fjmanzano.github.io/bioestadistica/inferencia-estad\%C3\%ADstica.html}{Explicación}
\end{enumerate}

\hypertarget{pregunta-test-6}{%
\section{Pregunta test}\label{pregunta-test-6}}

Elija la afirmación que pueda considerarse admisible al leer un estudio estadístico:

\begin{enumerate}
\def\labelenumi{\alph{enumi})}
\tightlist
\item
  Se estudió a una muestra en vez de a la población, para mayor precisión.
\item
  Se estudió a la población para obtener información sobre la muestra.
\item
  Se estudió a una muestra representativa de la población.
\item
  Se estudiaron todas las variables de la población.
\item
  Se observó a un individuo de cada variable.
\end{enumerate}

\hypertarget{soluciuxf3n-6}{%
\subsection{Solución}\label{soluciuxf3n-6}}

\begin{enumerate}
\def\labelenumi{\alph{enumi})}
\setcounter{enumi}{2}
\tightlist
\item
  \href{https://1fjmanzano.github.io/bioestadistica/me\%CC\%81todos-de-muestreo.html}{Explicación}
\end{enumerate}

\hypertarget{pregunta-test-7}{%
\section{Pregunta test}\label{pregunta-test-7}}

Elija la afirmación correcta:

\begin{enumerate}
\def\labelenumi{\alph{enumi})}
\tightlist
\item
  Los valores de cualquier variable deben ser agrupados en intervalos.
\item
  Las variables deben ofrecer valores que no se repitan en los diferentes individuos.
\item
  Las modalidades de una variable deben poder ser observadas en todos los individuos.
\item
  Los individuos pueden poseer diferentes modalidades de la misma variable.
\item
  Todo lo anterior es falso.
\end{enumerate}

\hypertarget{soluciuxf3n-7}{%
\subsection{Solución}\label{soluciuxf3n-7}}

\begin{enumerate}
\def\labelenumi{\alph{enumi})}
\setcounter{enumi}{2}
\tightlist
\item
  \href{https://1fjmanzano.github.io/bioestadistica/tipos-de-variables.html}{Explicación}
\end{enumerate}

\hypertarget{pregunta-test-8}{%
\section{Pregunta test}\label{pregunta-test-8}}

Elija la opción correcta.

\begin{enumerate}
\def\labelenumi{\alph{enumi})}
\tightlist
\item
  Un parámetro es algo calculado sobre cada individuo.
\item
  Un parámetro es calculado sobre la muestra.
\item
  Una variable se calcula sobre los parámetros de una población.
\item
  Un estadístico se calcula sobre la población.
\item
  Nada de lo anterior es correcto.
\end{enumerate}

\hypertarget{soluciuxf3n-8}{%
\subsection{Solución}\label{soluciuxf3n-8}}

\begin{enumerate}
\def\labelenumi{\alph{enumi})}
\setcounter{enumi}{4}
\tightlist
\item
  \href{https://1fjmanzano.github.io/bioestadistica/conceptos-previos.html}{Explicación}
\end{enumerate}

\hypertarget{pregunta-test-9}{%
\section{Pregunta test}\label{pregunta-test-9}}

Disponemos de la distribución de edades de los individuos de una población. El número de ellos que no es mayor de edad, es:

\begin{enumerate}
\def\labelenumi{\alph{enumi})}
\tightlist
\item
  Una frecuencia relativa.
\item
  Una frecuencia absoluta.
\item
  Una frecuencia acumulada.
\item
  Una variable numérica.
\item
  Una variable cualitativa.
\end{enumerate}

\hypertarget{soluciuxf3n-9}{%
\subsection{Solución}\label{soluciuxf3n-9}}

\begin{enumerate}
\def\labelenumi{\alph{enumi})}
\setcounter{enumi}{2}
\tightlist
\item
  \href{https://1fjmanzano.github.io/bioestadistica/tablas-de-frecuencias.html}{Explicación}
\end{enumerate}

\hypertarget{pregunta-test-10}{%
\section{Pregunta test}\label{pregunta-test-10}}

Conocemos la distribución de estudiantes entre las distintas facultades del campus Viriato. El número de estudiantes de Enfermería es:

\begin{enumerate}
\def\labelenumi{\alph{enumi})}
\tightlist
\item
  Una frecuencia relativa.
\item
  Una frecuencia absoluta.
\item
  Una frecuencia acumulada.
\item
  Un porcentaje.
\item
  Una variable cualitativa.
\end{enumerate}

\hypertarget{soluciuxf3n-10}{%
\subsection{Solución}\label{soluciuxf3n-10}}

\begin{enumerate}
\def\labelenumi{\alph{enumi})}
\setcounter{enumi}{1}
\tightlist
\item
  \href{https://1fjmanzano.github.io/bioestadistica/tablas-de-frecuencias.html}{Explicación}
\end{enumerate}

\hypertarget{pregunta-test-11}{%
\section{Pregunta test}\label{pregunta-test-11}}

Se llama parámetro a:

\begin{enumerate}
\def\labelenumi{\alph{enumi})}
\tightlist
\item
  Una función de valor numérico definida sobre alguna característica observable en los individuos de una población.
\item
  Una función definida sobre los valores numéricos de una muestra.
\item
  Cualquier variable observable de una población
\item
  Las variables numéricas de la muestra
\item
  Cualquier función sobre las variables observadas
\end{enumerate}

\hypertarget{soluciuxf3n-11}{%
\subsection{Solución}\label{soluciuxf3n-11}}

\begin{enumerate}
\def\labelenumi{\alph{enumi})}
\tightlist
\item
  \href{https://1fjmanzano.github.io/bioestadistica/conceptos-previos.html}{Explicación}
\end{enumerate}

\hypertarget{pregunta-test-12}{%
\section{Pregunta test}\label{pregunta-test-12}}

El grado de satisfacción (poco/regular/mucho) con la política española la trataría como:

\begin{enumerate}
\def\labelenumi{\alph{enumi})}
\tightlist
\item
  una variable cualitativa nominal.
\item
  una variable cuantitativa discreta.
\item
  una variable cualitativa ordinal.
\item
  una variable numérica continua.
\item
  ninguna de las anteriores es correcta.
\end{enumerate}

\hypertarget{soluciuxf3n-12}{%
\subsection{Solución}\label{soluciuxf3n-12}}

\begin{enumerate}
\def\labelenumi{\alph{enumi})}
\setcounter{enumi}{2}
\tightlist
\item
  \href{https://1fjmanzano.github.io/bioestadistica/tipos-de-variables.html}{Explicación}
\end{enumerate}

\hypertarget{pregunta-test-13}{%
\section{Pregunta test}\label{pregunta-test-13}}

Con respecto a la modalidades de una variable cualquiera:

\begin{enumerate}
\def\labelenumi{\alph{enumi})}
\tightlist
\item
  Pueden siempre agruparse en clases.
\item
  Deben formar un sistema exhaustivo.
\item
  No pueden agruparse en intervalos.
\item
  No tienen porqué formar un sistema excluyente.
\item
  Solo dos son correctas.
\end{enumerate}

\hypertarget{soluciuxf3n-13}{%
\subsection{Solución}\label{soluciuxf3n-13}}

\begin{enumerate}
\def\labelenumi{\alph{enumi})}
\setcounter{enumi}{1}
\tightlist
\item
  \href{https://1fjmanzano.github.io/bioestadistica/tipos-de-variables.html}{Explicación}
\end{enumerate}

\hypertarget{pregunta-test-14}{%
\section{Pregunta test}\label{pregunta-test-14}}

Cuando hablamos de número de cumpleaños que ha tenido una persona estamos ante:

\begin{enumerate}
\def\labelenumi{\alph{enumi})}
\tightlist
\item
  Una variable cualitativa ordinal.
\item
  Una variable cualitativa nominal.
\item
  Una variable cuantitativa discreta.
\item
  Una variable cuantitativa continua.
\item
  El número de cumpleaños no es una variable.
\end{enumerate}

\hypertarget{soluciuxf3n-14}{%
\subsection{Solución}\label{soluciuxf3n-14}}

\begin{enumerate}
\def\labelenumi{\alph{enumi})}
\setcounter{enumi}{2}
\tightlist
\item
  \href{https://1fjmanzano.github.io/bioestadistica/tipos-de-variables.html}{Explicación}
\end{enumerate}

\hypertarget{pregunta-test-15}{%
\section{Pregunta test}\label{pregunta-test-15}}

Las frecuencias acumuladas tienen sentido para:

\begin{enumerate}
\def\labelenumi{\alph{enumi})}
\tightlist
\item
  Variables ordinales
\item
  Variables numéricas
\item
  Variables nominales
\item
  Todas son correctas.
\item
  Las opciones a) y b) son correctas.
\end{enumerate}

\hypertarget{soluciuxf3n-15}{%
\subsection{Solución}\label{soluciuxf3n-15}}

\begin{enumerate}
\def\labelenumi{\alph{enumi})}
\setcounter{enumi}{4}
\tightlist
\item
  \href{https://1fjmanzano.github.io/bioestadistica/tablas-de-frecuencias.html}{Explicación}
\end{enumerate}

\hypertarget{pregunta-test-16}{%
\section{Pregunta test}\label{pregunta-test-16}}

Disponemos de la distribución de edades de los individuos de una población. El número de ellos que tiene dos o menos hijos es:

\begin{enumerate}
\def\labelenumi{\alph{enumi})}
\tightlist
\item
  Una variable cualitativa.
\item
  Una variable numérica.
\item
  Una frecuencia acumulada.
\item
  Son correctas a) y b)
\item
  Ninguna es correcta.
\end{enumerate}

\hypertarget{soluciuxf3n-16}{%
\subsection{Solución}\label{soluciuxf3n-16}}

\begin{enumerate}
\def\labelenumi{\alph{enumi})}
\setcounter{enumi}{2}
\tightlist
\item
  \href{https://1fjmanzano.github.io/bioestadistica/tablas-de-frecuencias.html}{Explicación}
\end{enumerate}

\hypertarget{anuxe1lisis-descriptivo-y-gruxe1fico-de-datos-cuantitativos}{%
\chapter{Análisis Descriptivo y Gráfico de datos cuantitativos}\label{anuxe1lisis-descriptivo-y-gruxe1fico-de-datos-cuantitativos}}

En este capítulo se resolverán problemas relativos a:

\begin{itemize}
\tightlist
\item
  Medidas de tendencia central: Media, Moda, Mediana.
\item
  Medidas de dispersión: Recorrido, Varianza, Desviación típica, Coeficiente de variación, Recorrido intercuartílico. Error estándar.
\item
  Representaciones gráficas: Diagrama de barras, Pictogramas, Cartogramas,
\end{itemize}

\hypertarget{pregunta-test-17}{%
\section{Pregunta test}\label{pregunta-test-17}}

Cuál de las siguientes medidas define mejor la tendencia central de los datos: 5 , 4, 42, 4, 6

\begin{enumerate}
\def\labelenumi{\alph{enumi})}
\tightlist
\item
  La mediana.
\item
  La media.
\item
  El sesgo
\item
  El rango.
\item
  La proporción.
\end{enumerate}

\hypertarget{soluciuxf3n-17}{%
\subsection{Solución}\label{soluciuxf3n-17}}

\begin{enumerate}
\def\labelenumi{\alph{enumi})}
\tightlist
\item
  \href{https://1fjmanzano.github.io/bioestadistica/medidas-de-posicio\%CC\%81n-dispersio\%CC\%81n-y-forma.html\#medidas-de-posicio\%CC\%81n-centrales}{Explicación}
\end{enumerate}

\hypertarget{pregunta-test-18}{%
\section{Pregunta test}\label{pregunta-test-18}}

Los diagramas de sectores son muy útiles para comparar:

\begin{enumerate}
\def\labelenumi{\alph{enumi})}
\tightlist
\item
  Dos variables cualitativas en una población.
\item
  Dos variables cuantitativas en una población.
\item
  Una variable cualitativa en dos poblaciones.
\item
  Una variable cuantitativa en dos poblaciones.
\item
  Una variable cuantitativa con otra cualitativa.
\end{enumerate}

\hypertarget{soluciuxf3n-18}{%
\subsection{Solución}\label{soluciuxf3n-18}}

\begin{enumerate}
\def\labelenumi{\alph{enumi})}
\setcounter{enumi}{2}
\tightlist
\item
  \href{https://1fjmanzano.github.io/bioestadistica/diagramas-de-barras-y-sectores.html}{Explicación}
\end{enumerate}

\hypertarget{pregunta-test-19}{%
\section{Pregunta test}\label{pregunta-test-19}}

En cuanto a la presentación ordenada del estudio de una variable aislada:

\begin{enumerate}
\def\labelenumi{\alph{enumi})}
\tightlist
\item
  Lo más informativo es mostrar las medidas de tendencia central.
\item
  Lo más informativo es mostrar las medidas de dispersión.
\item
  Se deben presentar todos los valores observados de la variable, uno a uno, de menor a mayor.
\item
  Las representaciones gráficas dan más información que las tablas de frecuencia.
\item
  A veces no tiene sentido usar frecuencias acumuladas.
\end{enumerate}

\hypertarget{soluciuxf3n-19}{%
\subsection{Solución}\label{soluciuxf3n-19}}

\begin{enumerate}
\def\labelenumi{\alph{enumi})}
\setcounter{enumi}{4}
\tightlist
\item
  \href{https://1fjmanzano.github.io/bioestadistica/otros-gra\%CC\%81ficos.html}{Explicación}
\end{enumerate}

\hypertarget{pregunta-test-20}{%
\section{Pregunta test}\label{pregunta-test-20}}

En las representaciones gráficas de variables cualitativas, la regla fundamental a tener en cuenta es:

\begin{enumerate}
\def\labelenumi{\alph{enumi})}
\tightlist
\item
  Las alturas en cada modalidad son proporcionales al valor de la variable.
\item
  Las áreas para cada modalidad son proporcionales al valor de la variable.
\item
  Las áreas para cada modalidad son proporcionales a las frecuencias acumuladas.
\item
  Las áreas para cada modalidad son proporcionales a las frecuencias absolutas o relativas.
\item
  Las alturas para cada modalidad son proporcionales a las frecuencias acumuladas.
\end{enumerate}

\hypertarget{soluciuxf3n-20}{%
\subsection{Solución}\label{soluciuxf3n-20}}

\begin{enumerate}
\def\labelenumi{\alph{enumi})}
\setcounter{enumi}{3}
\tightlist
\item
  \href{https://1fjmanzano.github.io/bioestadistica/diagramas-de-barras-y-sectores.html}{Explicación}
\end{enumerate}

\hypertarget{pregunta-test-21}{%
\section{Pregunta test}\label{pregunta-test-21}}

Entre las representaciones gráficas para variables cualitativas tenemos:

\begin{enumerate}
\def\labelenumi{\alph{enumi})}
\tightlist
\item
  Histogramas.
\item
  Diagramas integrales.
\item
  Diagramas diferenciales.
\item
  Diagramas de cajas y bigotes.
\item
  Nada de lo anterior.
\end{enumerate}

\hypertarget{soluciuxf3n-21}{%
\subsection{Solución}\label{soluciuxf3n-21}}

\begin{enumerate}
\def\labelenumi{\alph{enumi})}
\setcounter{enumi}{3}
\tightlist
\item
  \href{https://1fjmanzano.github.io/bioestadistica/diagramas-de-barras-y-sectores.html}{Explicación}
\end{enumerate}

\hypertarget{pregunta-test-22}{%
\section{Pregunta test}\label{pregunta-test-22}}

De los siguientes conceptos indique el que no tenga sentido:

\begin{enumerate}
\def\labelenumi{\alph{enumi})}
\tightlist
\item
  Diagrama de barras para la variable ``Grupo sanguíneo''
\item
  Pictograma para la variable ``Altura''
\item
  Diagrama integral para la variable ``Nivel de colesterol''
\item
  Diagrama de sectores para la variable ``Sexo''
\item
  Histograma para la variable ``Peso''
\end{enumerate}

\hypertarget{soluciuxf3n-22}{%
\subsection{Solución}\label{soluciuxf3n-22}}

\begin{enumerate}
\def\labelenumi{\alph{enumi})}
\setcounter{enumi}{1}
\tightlist
\item
  \href{https://1fjmanzano.github.io/bioestadistica/otros-gra\%CC\%81ficos.html}{Explicación}
\end{enumerate}

\hypertarget{pregunta-test-23}{%
\section{Pregunta test}\label{pregunta-test-23}}

Si queremos representar gráficamente los porcentajes de una variable cuantitativa continua debemos usar:

\begin{enumerate}
\def\labelenumi{\alph{enumi})}
\tightlist
\item
  Pictogramas
\item
  Diagrama de barras
\item
  Diagrama diferencial acumulado
\item
  Histograma
\item
  No existe gráfica posible
\end{enumerate}

\hypertarget{soluciuxf3n-23}{%
\subsection{Solución}\label{soluciuxf3n-23}}

\begin{enumerate}
\def\labelenumi{\alph{enumi})}
\setcounter{enumi}{3}
\tightlist
\item
  \href{https://1fjmanzano.github.io/bioestadistica/histogramas.html}{Explicación}
\end{enumerate}

\hypertarget{pregunta-test-24}{%
\section{Pregunta test}\label{pregunta-test-24}}

Los gráficos indicados para variables cualitativas son:

\begin{enumerate}
\def\labelenumi{\alph{enumi})}
\item
  Los diagramas de barras y los histogramas
\item
  Los diagramas de barras, los de sectores y los pictogramas
\item
  Los histogramas y pictogramas
\item
  Sólo los diagramas de barras
\item
  Los diagramas integrales
\item
  \href{https://1fjmanzano.github.io/bioestadistica/diagramas-de-barras-y-sectores.html}{Explicación}
\end{enumerate}

\hypertarget{pregunta-test-25}{%
\section{Pregunta test}\label{pregunta-test-25}}

¿Qué gráfico elegirías para representar una las respuestas a una encuesta sobre el número de hijos que tiene la población?

\begin{enumerate}
\def\labelenumi{\alph{enumi})}
\tightlist
\item
  Histograma
\item
  Diagrama de sectores
\item
  Pictograma
\item
  Diagrama de Barras
\item
  Ninguna de las anteriores
\end{enumerate}

\hypertarget{soluciuxf3n-24}{%
\subsection{Solución}\label{soluciuxf3n-24}}

\begin{enumerate}
\def\labelenumi{\alph{enumi})}
\setcounter{enumi}{3}
\tightlist
\item
  \href{https://1fjmanzano.github.io/bioestadistica/diagramas-de-barras-y-sectores.html}{Explicación}
\end{enumerate}

\hypertarget{anuxe1lisis-inferencial.-aplicaciones.}{%
\chapter{Análisis Inferencial. Aplicaciones.}\label{anuxe1lisis-inferencial.-aplicaciones.}}

En este capítulo se resolverán problemas relativos a:

\begin{itemize}
\tightlist
\item
  Objetivos del estudio, hipótesis de trabajo e hipótesis estadísticas
\item
  Importancia de las distribuciones de probabilidad en el trabajo práctico
\item
  Estimación puntual y por intervalo
\item
  Verificación de las hipótesis de trabajo: contraste de hipótesis
\end{itemize}

\hypertarget{regresiuxf3n-y-correlaciuxf3n.}{%
\chapter{Regresión y correlación.}\label{regresiuxf3n-y-correlaciuxf3n.}}

En este capítulo se resolverán problemas relativos a:

\begin{itemize}
\tightlist
\item
  Introducción a la regresión y correlación
\item
  Estudio de la representatividad de la recta de regresión
\item
  Otros modelos de regresión
\item
  Correlación
\end{itemize}

\hypertarget{tablas-de-contingencia.}{%
\chapter{Tablas de contingencia.}\label{tablas-de-contingencia.}}

En este capítulo se resolverán problemas relativos a:

\begin{itemize}
\tightlist
\item
  Contrastes de asociación y homogeneidad en tablas bifactoriales
\item
  Coeficientes de asociación
\end{itemize}

\hypertarget{medidas-de-importancia-cluxednica.}{%
\chapter{Medidas de importancia clínica.}\label{medidas-de-importancia-cluxednica.}}

En este capítulo se resolverán problemas relativos a:

\begin{itemize}
\tightlist
\item
  Diferencias entre Proporción, Tasa, Razón, odds.
\item
  Medidas de asociación en tablas 2x2. Riesgo Relativo. Riesgo Absolutos. Odds-Ratio.
\item
  Indicadores estadísticos básicos para evaluar el desempeño de un procedimiento diagnóstico: Sensibilidad y Especificidad. Probabilidades pre y post prueba.
\end{itemize}

  \bibliography{book.bib,packages.bib}

\end{document}
