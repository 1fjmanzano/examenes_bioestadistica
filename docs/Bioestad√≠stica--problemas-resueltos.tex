% Options for packages loaded elsewhere
\PassOptionsToPackage{unicode}{hyperref}
\PassOptionsToPackage{hyphens}{url}
%
\documentclass[
]{book}
\usepackage{lmodern}
\usepackage{amsmath}
\usepackage{ifxetex,ifluatex}
\ifnum 0\ifxetex 1\fi\ifluatex 1\fi=0 % if pdftex
  \usepackage[T1]{fontenc}
  \usepackage[utf8]{inputenc}
  \usepackage{textcomp} % provide euro and other symbols
  \usepackage{amssymb}
\else % if luatex or xetex
  \usepackage{unicode-math}
  \defaultfontfeatures{Scale=MatchLowercase}
  \defaultfontfeatures[\rmfamily]{Ligatures=TeX,Scale=1}
\fi
% Use upquote if available, for straight quotes in verbatim environments
\IfFileExists{upquote.sty}{\usepackage{upquote}}{}
\IfFileExists{microtype.sty}{% use microtype if available
  \usepackage[]{microtype}
  \UseMicrotypeSet[protrusion]{basicmath} % disable protrusion for tt fonts
}{}
\makeatletter
\@ifundefined{KOMAClassName}{% if non-KOMA class
  \IfFileExists{parskip.sty}{%
    \usepackage{parskip}
  }{% else
    \setlength{\parindent}{0pt}
    \setlength{\parskip}{6pt plus 2pt minus 1pt}}
}{% if KOMA class
  \KOMAoptions{parskip=half}}
\makeatother
\usepackage{xcolor}
\IfFileExists{xurl.sty}{\usepackage{xurl}}{} % add URL line breaks if available
\IfFileExists{bookmark.sty}{\usepackage{bookmark}}{\usepackage{hyperref}}
\hypersetup{
  pdftitle={Bioestadística: problemas resueltos},
  pdfauthor={Javier Manzano},
  hidelinks,
  pdfcreator={LaTeX via pandoc}}
\urlstyle{same} % disable monospaced font for URLs
\usepackage{longtable,booktabs}
\usepackage{calc} % for calculating minipage widths
% Correct order of tables after \paragraph or \subparagraph
\usepackage{etoolbox}
\makeatletter
\patchcmd\longtable{\par}{\if@noskipsec\mbox{}\fi\par}{}{}
\makeatother
% Allow footnotes in longtable head/foot
\IfFileExists{footnotehyper.sty}{\usepackage{footnotehyper}}{\usepackage{footnote}}
\makesavenoteenv{longtable}
\usepackage{graphicx}
\makeatletter
\def\maxwidth{\ifdim\Gin@nat@width>\linewidth\linewidth\else\Gin@nat@width\fi}
\def\maxheight{\ifdim\Gin@nat@height>\textheight\textheight\else\Gin@nat@height\fi}
\makeatother
% Scale images if necessary, so that they will not overflow the page
% margins by default, and it is still possible to overwrite the defaults
% using explicit options in \includegraphics[width, height, ...]{}
\setkeys{Gin}{width=\maxwidth,height=\maxheight,keepaspectratio}
% Set default figure placement to htbp
\makeatletter
\def\fps@figure{htbp}
\makeatother
\setlength{\emergencystretch}{3em} % prevent overfull lines
\providecommand{\tightlist}{%
  \setlength{\itemsep}{0pt}\setlength{\parskip}{0pt}}
\setcounter{secnumdepth}{5}
\usepackage{booktabs}
\usepackage[spanish]{babel}
\decimalpoint
\selectlanguage{spanish}
\ifluatex
  \usepackage{selnolig}  % disable illegal ligatures
\fi
\usepackage[]{natbib}
\bibliographystyle{apalike}

\title{Bioestadística: problemas resueltos}
\author{Javier Manzano}
\date{2023-10-05}

\begin{document}
\maketitle

{
\setcounter{tocdepth}{1}
\tableofcontents
}
\hypertarget{introducciuxf3n}{%
\chapter{Introducción}\label{introducciuxf3n}}

En estas páginas encontrarás problemas resueltos tipo examen de Bioestadística para la asignatura en Grados de Ciencias de la Salud (Enfermería, Fisioterapia, Farmacia, etc.)

Estas páginas son un complemento al \href{https://1fjmanzano.github.io/bioestadistica/}{Curso de Bioestadística} que incluye prácticas con Excel©.

En temario sobre el que basamos esta colección de problemas es el de la asignatura de la Universidad de Salamanca que incluye los siguientes bloques temáticos:

1.- Planteamiento de una investigación: Anatomía y Fisiología de la investigación

2..-Análisis Descriptivo y Gráfico de datos cuantitativos

3.- Análisis Inferencial. Aplicaciones.

4.-Regresión y correlación.

5.-Tablas de contingencia.

6.- Medidas de importancia clínica.

\hypertarget{planteamiento-de-una-investigaciuxf3n-anatomuxeda-y-fisiologuxeda-de-la-investigaciuxf3n}{%
\chapter{Planteamiento de una investigación: Anatomía y Fisiología de la investigación}\label{planteamiento-de-una-investigaciuxf3n-anatomuxeda-y-fisiologuxeda-de-la-investigaciuxf3n}}

En este capítulo se resolverán problemas relativos a:

\begin{itemize}
\tightlist
\item
  Diseño de una investigación
\item
  Métodos de muestreo
\item
  Métodos de recolección de datos
\item
  Variables y Escalas de Medida
\item
  Errores en la Investigación
\end{itemize}

\hypertarget{anuxe1lisis-descriptivo-y-gruxe1fico-de-datos-cuantitativos}{%
\chapter{Análisis Descriptivo y Gráfico de datos cuantitativos}\label{anuxe1lisis-descriptivo-y-gruxe1fico-de-datos-cuantitativos}}

En este capítulo se resolverán problemas relativos a:

\begin{itemize}
\tightlist
\item
  Medidas de tendencia central: Media, Moda, Mediana.
\item
  Medidas de dispersión: Recorrido, Varianza, Desviación típica, Coeficiente de variación, Recorrido intercuartílico. Error estándar.
\item
  Representaciones gráficas: Diagrama de barras, Pictogramas, Cartogramas,
\end{itemize}

\hypertarget{anuxe1lisis-inferencial.-aplicaciones.}{%
\chapter{Análisis Inferencial. Aplicaciones.}\label{anuxe1lisis-inferencial.-aplicaciones.}}

En este capítulo se resolverán problemas relativos a:

\begin{itemize}
\tightlist
\item
  Objetivos del estudio, hipótesis de trabajo e hipótesis estadísticas
\item
  Importancia de las distribuciones de probabilidad en el trabajo práctico
\item
  Estimación puntual y por intervalo
\item
  Verificación de las hipótesis de trabajo: contraste de hipótesis
\end{itemize}

\hypertarget{regresiuxf3n-y-correlaciuxf3n.}{%
\chapter{Regresión y correlación.}\label{regresiuxf3n-y-correlaciuxf3n.}}

En este capítulo se resolverán problemas relativos a:

\begin{itemize}
\tightlist
\item
  Introducción a la regresión y correlación
\item
  Estudio de la representatividad de la recta de regresión
\item
  Otros modelos de regresión
\item
  Correlación
\end{itemize}

\hypertarget{tablas-de-contingencia.}{%
\chapter{Tablas de contingencia.}\label{tablas-de-contingencia.}}

En este capítulo se resolverán problemas relativos a:

\begin{itemize}
\tightlist
\item
  Contrastes de asociación y homogeneidad en tablas bifactoriales
\item
  Coeficientes de asociación
\end{itemize}

\hypertarget{medidas-de-importancia-cluxednica.}{%
\chapter{Medidas de importancia clínica.}\label{medidas-de-importancia-cluxednica.}}

En este capítulo se resolverán problemas relativos a:

\begin{itemize}
\tightlist
\item
  Diferencias entre Proporción, Tasa, Razón, odds.
\item
  Medidas de asociación en tablas 2x2. Riesgo Relativo. Riesgo Absolutos. Odds-Ratio.
\item
  Indicadores estadísticos básicos para evaluar el desempeño de un procedimiento diagnóstico: Sensibilidad y Especificidad. Probabilidades pre y post prueba.
\end{itemize}

  \bibliography{book.bib,packages.bib}

\end{document}
